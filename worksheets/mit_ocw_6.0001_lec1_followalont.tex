\documentclass[11pt]{article}
\usepackage[margin=0.75in]{geometry}
\usepackage{tikz}
\usepackage{tcolorbox}
\usepackage{enumitem}
\usepackage{fancyhdr}
\usepackage{array}
\usepackage{tabularx}
\usepackage{xcolor}
\usepackage{graphicx}
\usepackage{amssymb}
\usepackage{amsmath}
\setlength{\headheight}{14pt}

% Colors
\definecolor{mitred}{RGB}{163, 31, 52}
\definecolor{lightgray}{RGB}{245, 245, 245}
\definecolor{darkblue}{RGB}{0, 51, 102}

% Custom blank command
\newcommand{\blank}[1]{\underline{\hspace{#1}}}

% Header/Footer
\pagestyle{fancy}
\fancyhf{}
\lhead{\textbf{MIT 6.0001} | Lecture 1: What is Computation?}
\rhead{Follow-Along Worksheet}
\lfoot{Name: \underline{\hspace{2in}} \quad Period: \underline{\hspace{0.5in}}}
\rfoot{Page \thepage}
\renewcommand{\headrulewidth}{0.4pt}
\renewcommand{\footrulewidth}{0.4pt}

\begin{document}

\begin{center}
\textbf{\LARGE Introduction to Computer Science and Programming in Python}\\[0.5em]
\textbf{\Large Lecture 1: What is Computation?}\\[0.3em]
\textit{MIT OpenCourseWare 6.0001 | Fall 2016 | Instructor: Ana Bell}
\end{center}

\vspace{0.5em}

%% LEARNING OBJECTIVES BOX
\begin{tcolorbox}[colback=white, colframe=darkblue, title=\textbf{Learning Objectives}, fonttitle=\color{white}]
By the end of this lecture, you will be able to:
\begin{itemize}[leftmargin=*, itemsep=2pt]
    \item Distinguish between \textbf{declarative} and \textbf{imperative} knowledge
    \item Identify the three components that define an algorithm
    \item Describe the basic architecture of a stored-program computer
    \item Differentiate between syntax, static semantics, and semantics
    \item Use Python's scalar object types (int, float, bool, NoneType)
    \item Write and evaluate simple expressions using operators
    \item Create variable bindings and understand how assignment works in memory
\end{itemize}
\end{tcolorbox}

\vspace{0.5em}

%% VOCABULARY BOX
\begin{tcolorbox}[colback=lightgray, colframe=gray, title=\textbf{Key Vocabulary}]
\begin{tabularx}{\textwidth}{>{\bfseries}l X}
Algorithm & A recipe for computation: a sequence of steps, flow of control, and a stopping condition \\[3pt]
Declarative Knowledge & Statements of fact (``what is true'') \\[3pt]
Imperative Knowledge & A recipe or procedure (``how to'' do something) \\[3pt]
Syntax & The rules governing what forms are legal in a language \\[3pt]
Semantics & The meaning of a syntactically correct expression \\[3pt]
Binding & Associating a variable name with a value in memory \\[3pt]
Expression & A combination of objects and operators that evaluates to a value \\
\end{tabularx}
\end{tcolorbox}

\vspace{1em}

%% SECTION 1: Course Introduction
\begin{tcolorbox}[colback=mitred!5, colframe=mitred, title=\textbf{[0:00--4:00] Course Introduction \& The Three Pillars}]
\end{tcolorbox}

\vspace{-0.5em}

\begin{enumerate}[leftmargin=*, label=\arabic*.]
    \item This is a \textbf{fast-paced course}. The key advice for beginners is:\\ 
    \blank{1.5in}, \blank{1.5in}, \blank{1.5in}!
    
    \item All learning in programming depends on three interconnected pillars:
    \begin{itemize}[itemsep=4pt]
        \item \blank{2in} of concepts
        \item \blank{2in} skill
        \item \blank{2in} solving
    \end{itemize}
    
    \item Which of these pillars does \textit{practice} support? \blank{2in}
\end{enumerate}

\vspace{0.8em}

%% SECTION 2: What Does a Computer Do?
\begin{tcolorbox}[colback=mitred!5, colframe=mitred, title=\textbf{[7:30--10:30] What Does a Computer Do?}]
\end{tcolorbox}

\vspace{-0.5em}

\begin{enumerate}[resume, leftmargin=*, label=\arabic*.]
    \item Fundamentally, a computer does two things:
    \begin{itemize}[itemsep=4pt]
        \item Performs \blank{2in}
        \item \blank{2in} results
    \end{itemize}
    
    \item \textbf{Critical principle:} A computer only knows \blank{2.5in}.
    
    \item \textbf{Two types of knowledge:}
    \begin{itemize}[itemsep=4pt]
        \item \textbf{Declarative:} Statements of \blank{1in}. Example: ``The square root of a number $x$ is $y$ such that $y \times y = x$.''
        \item \textbf{Imperative:} A \blank{1.5in} for how to find something. Example: A step-by-step method to compute a square root.
    \end{itemize}
    
    \item Can a computer directly use declarative knowledge to solve problems? \blank{1in}\\[4pt]
    Why or why not? \blank{4.5in}
\end{enumerate}

\vspace{0.8em}

%% SECTION 3: Algorithms
\begin{tcolorbox}[colback=mitred!5, colframe=mitred, title=\textbf{[12:50--15:30] What is an Algorithm?}]
\end{tcolorbox}

\vspace{-0.5em}

\begin{enumerate}[resume, leftmargin=*, label=\arabic*.]
    \item An algorithm (or ``recipe'') has \textbf{three required components}:
    \begin{enumerate}[label=(\alph*), itemsep=6pt]
        \item A sequence of \blank{1.5in} \blank{1.5in}
        \item \blank{1.5in} of \blank{1.5in} (decisions, repetition)
        \item A \blank{1.5in} condition (when to stop)
    \end{enumerate}
    
    \item Consider the square root algorithm shown in lecture. Complete the trace:
    
    \begin{center}
    \begin{tabular}{|c|c|c|}
    \hline
    \textbf{Iteration} & \textbf{Guess ($g$)} & \textbf{$g \times g$} \\
    \hline
    Start & 3 & 9 \\
    \hline
    1 & \blank{0.6in} & \blank{0.6in} \\
    \hline
    2 & \blank{0.6in} & \blank{0.6in} \\
    \hline
    \end{tabular}
    \end{center}
    (Finding $\sqrt{16}$ starting with guess = 3, using $g_{\text{new}} = \frac{g + \frac{x}{g}}{2}$)
\end{enumerate}

\vspace{0.8em}

%% SECTION 4: Machine Architecture
\begin{tcolorbox}[colback=mitred!5, colframe=mitred, title=\textbf{[16:30--22:00] Computers as Machines}]
\end{tcolorbox}

\vspace{-0.5em}

\begin{enumerate}[resume, leftmargin=*, label=\arabic*.]
    \item A \textbf{fixed program computer} (like a calculator) can only do \blank{2in}.
    
    \item A \textbf{stored program computer} treats instructions as \blank{1.5in}. This means the same machine can run \blank{2in} programs.
    
    \item The four main components of basic machine architecture:
    \begin{itemize}[itemsep=4pt]
        \item \blank{1.5in} (stores data and instructions)
        \item \blank{2in} (ALU --- does math)
        \item \blank{1.5in} Unit (manages execution order)
        \item Input / \blank{1in}
    \end{itemize}
    
    \item Alan Turing proved that \blank{1in} primitives can compute \blank{1.5in}.
    
    \item \textbf{Turing completeness:} Anything computable in one language\\is \blank{2.5in}.
\end{enumerate}

\newpage

%% SECTION 5: Language Structure
\begin{tcolorbox}[colback=mitred!5, colframe=mitred, title=\textbf{[22:00--28:00] Aspects of Programming Languages}]
\end{tcolorbox}

\vspace{-0.5em}

\begin{enumerate}[resume, leftmargin=*, label=\arabic*.]
    \item Every programming language has these aspects:
    
    \vspace{0.5em}
    \small
    \begin{tabularx}{\textwidth}{|>{\bfseries}p{1.5in}|X|}
    \hline
    Aspect & Definition \\
    \hline
    Primitive constructs & The basic \blank{1.2in} of the language \\
    \hline
    Syntax & Rules for which \blank{1.2in} are \blank{0.8in} \\
    \hline
    Static semantics & Which valid strings have \blank{1.2in} \\
    \hline
    Semantics & The \blank{1.2in} of a valid expression \\
    \hline
    \end{tabularx}
    \normalsize
    
    \item \textbf{Example:} ``\texttt{"hi"5}'' is a \blank{1.5in} error in Python.
    
    \item \textbf{Example:} ``I are hungry'' is syntactically valid English but has a \blank{2in} error.
    
    \item Programs in Python have exactly \blank{1in} meaning. But that meaning might not match the programmer's \blank{1.5in}.
\end{enumerate}

\vspace{0.8em}

%% SECTION 6: Error Types
\begin{tcolorbox}[colback=lightgray, colframe=gray, title=\textbf{Error Taxonomy}]
\small
\begin{tabularx}{\textwidth}{|>{\bfseries}p{1.3in}|X|X|}
\hline
Error Type & What It Means & Example \\
\hline
Syntactic & \blank{1.2in} & Missing parenthesis \\
\hline
Static semantic & Legal form but \blank{1in} & \texttt{"hi" + 5} \\
\hline
Semantic/Logical & \blank{1.2in} but wrong result & Off-by-one loop \\
\hline
\end{tabularx}
\end{tcolorbox}

\vspace{1em}

%% SECTION 7: Python Objects
\begin{tcolorbox}[colback=mitred!5, colframe=mitred, title=\textbf{[28:57--33:00] Python Objects and Types}]
\end{tcolorbox}

\vspace{-0.5em}

\begin{enumerate}[resume, leftmargin=*, label=\arabic*.]
    \item A Python program is a sequence of \blank{1.5in} and \blank{1.5in}.
    
    \item Everything in Python is an \blank{1.5in}. Every object has a \blank{1in}.
    
    \item The type of an object determines what \blank{2in} can be performed on it.
    
    \item \textbf{Scalar types} (indivisible):
    \begin{itemize}[itemsep=4pt]
        \item \texttt{int}: \blank{2in} (e.g., \texttt{5}, \texttt{-3})
        \item \texttt{float}: \blank{2in} numbers (e.g., \texttt{3.14})
        \item \texttt{bool}: \blank{1.5in} values (\texttt{True} or \texttt{False})
        \item \texttt{NoneType}: Special type with only value \blank{1in}
    \end{itemize}
    
    \item To check an object's type, use \texttt{type(\blank{1in})}
    
    \item \textbf{Type conversion (casting):}
    \begin{itemize}[itemsep=4pt]
        \item \texttt{float(3)} returns \blank{1in}
        \item \texttt{int(3.9)} returns \blank{1in} (note: truncates, doesn't round!)
    \end{itemize}
\end{enumerate}

\vspace{0.8em}

%% SECTION 8: Expressions and Operators
\begin{tcolorbox}[colback=mitred!5, colframe=mitred, title=\textbf{[34:30--36:00] Expressions and Operators}]
\end{tcolorbox}

\vspace{-0.5em}

\begin{enumerate}[resume, leftmargin=*, label=\arabic*.]
    \item An \textbf{expression} combines \blank{1.5in} and \blank{1.5in} to produce a value.
    
    \item \textbf{Operators on \texttt{int} and \texttt{float}:}
    
    \vspace{0.3em}
    \small
    \begin{tabularx}{\textwidth}{|c|X|l|}
    \hline
    \textbf{Op} & \textbf{Operation} & \textbf{Example} \\
    \hline
    \texttt{+} & Addition & \texttt{3 + 2} $\rightarrow$ \texttt{5} \\
    \hline
    \texttt{-} & Subtraction & \texttt{5 - 1} $\rightarrow$ \texttt{4} \\
    \hline
    \texttt{*} & Multiplication & \texttt{2 * 3} $\rightarrow$ \texttt{6} \\
    \hline
    \texttt{/} & Division (returns \blank{0.6in}) & \texttt{5 / 2} $\rightarrow$ \blank{0.6in} \\
    \hline
    \texttt{//} & \blank{1in} division & \texttt{5 // 2} $\rightarrow$ \texttt{2} \\
    \hline
    \texttt{\%} & \blank{1in} (remainder) & \texttt{5 \% 2} $\rightarrow$ \blank{0.4in} \\
    \hline
    \texttt{**} & \blank{1in} (exponent) & \texttt{2 ** 3} $\rightarrow$ \texttt{8} \\
    \hline
    \end{tabularx}
    \normalsize
\end{enumerate}

\vspace{0.8em}

%% SECTION 9: Variables and Binding
\begin{tcolorbox}[colback=mitred!5, colframe=mitred, title=\textbf{[36:13--42:14] Variables and Assignment}]
\end{tcolorbox}

\vspace{-0.5em}

\begin{enumerate}[resume, leftmargin=*, label=\arabic*.]
    \item The \texttt{=} sign in Python means \blank{1.5in}, \textbf{not} mathematical equality.
    
    \item Assignment \textbf{binds} a \blank{1in} to a \blank{1in} in memory.
    
    \item When you write \texttt{pi = 3.14159}:
    \begin{enumerate}[label=(\alph*)]
        \item Python evaluates the \blank{1.5in} side first
        \item Then binds the name \texttt{pi} to that \blank{1in}
    \end{enumerate}
    
    \item Why is \texttt{radius = radius + 1} valid in Python but not in math?\\[6pt]
    \blank{5.5in}
    
    \item \textbf{Trace through this code:}
    \begin{verbatim}
    pi = 3.14
    radius = 2.2
    area = pi * (radius ** 2)
    radius = radius + 1
    \end{verbatim}
    
    After this code runs:
    \begin{itemize}[itemsep=4pt]
        \item \texttt{pi} = \blank{1in}
        \item \texttt{radius} = \blank{1in}
        \item \texttt{area} = \blank{1.5in}
    \end{itemize}
    
    \item \textbf{Critical insight:} Does \texttt{area} automatically update when \texttt{radius} changes?\\[4pt]
    \blank{1in}. The computer does not ``remember \blank{2in}.'' It only remembers \blank{1.5in}.
\end{enumerate}

\newpage

%% POST-VIDEO REFLECTION
\begin{tcolorbox}[colback=white, colframe=darkblue, title=\textbf{Post-Video Reflection}, fonttitle=\color{white}]
\end{tcolorbox}

\vspace{0.5em}

\textbf{Directions:} Answer the following questions in complete sentences.

\vspace{0.5em}

\begin{enumerate}[leftmargin=*, label=\textbf{R\arabic*.}]
    \item The lecture emphasizes that ``a computer only does what you tell it to do.'' Why is this principle so important for beginning programmers to understand? What misconceptions might this correct?
    
    \vspace{1.2in}
    
    \item Explain in your own words why a computer cannot directly use declarative knowledge (like ``the square root of 16 is 4'') to solve problems. What must a programmer provide instead?
    
    \vspace{1.2in}
    
    \item Consider this code:
    \begin{verbatim}
        price = 100
        tax_rate = 0.08
        total = price * (1 + tax_rate)
        price = 120
    \end{verbatim}
    A beginning programmer might expect \texttt{total} to automatically become \texttt{129.6} after the last line. Explain why it doesn't, using the memory-binding model from the lecture.
    
    \vspace{1.2in}
    
    \item The lecture introduces three types of errors: syntactic, static semantic, and semantic/logical. Which type do you think is hardest to find and fix? Why?
    
    \vspace{1.2in}
\end{enumerate}

\vspace{0.5em}

%% EXIT TICKET
\begin{tcolorbox}[colback=lightgray, colframe=darkblue, title=\textbf{Exit Ticket}]
In 2--3 sentences, explain what distinguishes a \textbf{stored-program computer} from a \textbf{fixed-program computer}, and why this distinction matters for programming.

\vspace{1in}
\end{tcolorbox}

\vspace{1em}

%% CONNECTION TO PROBLEM SET
\begin{tcolorbox}[colback=white, colframe=mitred, title=\textbf{Connection to Problem Set 0}]
Your first problem set asks you to write a program that:
\begin{enumerate}[itemsep=2pt]
    \item Takes user input (using \texttt{input()})
    \item Performs calculations (power and logarithm)
    \item Prints results (using \texttt{print()})
\end{enumerate}

\textbf{Concepts from today you'll need:}
\begin{itemize}[itemsep=2pt]
    \item Variable binding (\texttt{x = input("Enter x: ")})
    \item Type conversion (\texttt{int()}, \texttt{float()})
    \item Operators (\texttt{**} for power)
    \item The difference between typing in the shell vs.\ running a script
\end{itemize}
\end{tcolorbox}

\end{document}